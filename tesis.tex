\documentclass[12pt]{report}

% Paquetes comunes
\usepackage[utf8]{inputenc}
\usepackage[spanish]{babel}
\usepackage{amsmath, amssymb}
\usepackage{graphicx}
\usepackage{fancyhdr}
\usepackage{titlesec}
\usepackage{hyperref}
\usepackage{geometry}
\usepackage{amsthm}  % Carga el paquete necesario

\newtheorem{theorem}{Teorema}  % Define el entorno "theorem"
\geometry{a4paper, margin=2.5cm}

% Encabezados y pies de página
\pagestyle{fancy}
\fancyhf{}
\fancyhead[L]{\leftmark}
\fancyfoot[C]{\thepage}

% Título
\title{Título del Trabajo}
\author{Tu Nombre Completo\\ \small{Universidad o Institución}}
\date{\today}

\begin{document}

% Carátula
\begin{titlepage}
    \centering
    \includegraphics[width=0.2\textwidth]{imagenes/logo_UBA.png}\par\vspace{1cm} % Cambiar por el logo si se desea
    {\scshape\LARGE Universidad de Buenos Aires \par}
    \vspace{1cm}
    {\scshape\Large Facultad de Ciencias Exactas \par}
    {\scshape\Large Departamento de Matemática \par}
    \vspace{1.5cm}
    {\huge\bfseries Comportamiento asintótico del tiempo de covertura en arboles de Galton-Watson\par}

    \vspace{2cm}
    {\Large Joaquin E. Viera\par}
    \vspace{0.5cm}
    {\Large Directora: Inés Armendariz \par}
    {\Large Co-Director: Santiago Saglietti \par}

    \vfill
    {\large Fecha: \today\par}
\end{titlepage}

% Abstract
\chapter*{Resumen / Abstract}
\addcontentsline{toc}{chapter}{Resumen / Abstract}
Aquí va el resumen del trabajo. Puedes incluir objetivos, metodología, resultados y conclusiones más importantes.

\vspace{1cm}
\textbf{Palabras clave:} palabra1, palabra2, palabra3.

% Agradecimientos
\chapter*{Agradecimientos}
\addcontentsline{toc}{chapter}{Agradecimientos}
Aquí puedes agradecer a quienes colaboraron en el desarrollo del trabajo: familiares, profesores, instituciones, etc.

% Índice
\tableofcontents
\newpage

% Introducción
\chapter{Introducción}
\section{Contexto}
Describe el contexto general del tema tratado.

\section{Motivación}
Explica por qué elegiste este tema.

\section{Objetivos}
Menciona los objetivos generales y específicos.

\section{Estructura del documento}
Describe brevemente qué se trata en cada capítulo.

% Capítulo 1
\chapter{Presentación del modelo y preliminares}
\section{Concepto 1}
Explicación y fuentes.

\section{Concepto 2}
Más teoría relacionada.

% Capítulo 2
\chapter{Relación del cover timer con el branching process/GFF}
\section{Diseño del estudio}
Describe el enfoque.

\section{Procedimientos}
Explica cómo se llevó a cabo.

% Capítulo 3
\chapter{Resultado sobre el branching process}
\section{Resultado sobre la ultima generación}
Durante esta sección, para no sobrecargar de notación, dado un arbol de Galton-Watson $T$ que no se extingue 
vamos a considerar $\mathbb{P}(\cdot|T) = \mathbb{P}(\cdot)$, analogamente con la esperanza.
\begin{theorem}
Dado un GFF $\eta = (\eta_{v})_{v \in T_{n}}$, construido como antes. Entonces,
\begin{equation} 
\mathbb{E}[\max_{v \in L_n} \eta_v] = n\sqrt{2\log m} \, (1 + o(1)).
\end{equation}
\end{theorem}

\subsection{Cota superior}

Sea $\bar{Z}_n = \sum_{v \in L_n} \mathbf{1}_{S_v > (1 + \epsilon)x^* n}$, 
que cuenta la cantidad de vertices, en la $n$-th generación, 
se encuentran por arriba de $n x^*(1 + \epsilon)$. Aplicando el metodo del primer momento: 
tenemos, para todo $v \in L_n$,
\[
\mathbb{E} \bar{Z}_n = |L_n| \mathbb{P}(S_v > n(1 + \epsilon)x^*) \leq CW k^n e^{-n I((1 + \epsilon)x^*)},
\]
Donde aplicamos la desigualdad de Chebyshev en la ultima desigualdad y la definicion de $I$. Además, por 
la monotonia estricta de $I$, tenemos que $\mathbb{E} \bar{Z}_n \leq e^{-n c(\epsilon)}$, 
para algun $c(\epsilon) > 0$. Por lo tanto,
\begin{equation}
\mathbb{P}(M_n > (1 + \epsilon) n x^*) \leq \mathbb{E} [\bar{Z}_n] \leq CWe^{-c(\epsilon)n}.
\end{equation}
Por otro lado,
\begin{equation}
\mathbb{E}M_n \leq \mathbb{E}M_n\mathbf{1}_{M_n \geq 0} = \int_{0}^{\infty} \mathbb{P}(M_n > t) dt = 
\int_{0}^{(1+\epsilon)nx^*} \mathbb{P}(M_n > t) dt + \int_{(1+\epsilon)nx^*}^{\infty} \mathbb{P}(M_n > t) dt 
\end{equation}
Luego, usando la cota de 4.2 en el segundo integrando de 4.3 e integrando, llegamos a que,
\begin{equation}
\mathbb{E}M_n \leq nx^*(1+\epsilon) + nx^* \frac{CWe^{-2nI(x^*)\epsilon}}{2nI(x^*)}.
\end{equation}
Para todo $\epsilon > 0$. Haciendo $\epsilon \to 0$ obtenemos la cota superior.

\subsection{Cota inferior}



\section{Resultado sobre todo el arbol}
Comparación con literatura o hipótesis.

% Capítulo 4
\chapter{Conclusiones}
\section{Conclusiones generales}
Resumen de hallazgos.

\section{Trabajo futuro}
Ideas para desarrollos posteriores.

% Bibliografía
\begin{thebibliography}{9}
\bibitem{autor2020}
Apellido, Nombre. \textit{Título del libro o artículo}. Editorial, Año.

\bibitem{otroautor2021}
Otro Apellido, Otro Nombre. \textit{Otro título}. Otra Editorial, 2021.
\end{thebibliography}

\end{document}
